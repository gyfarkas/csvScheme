\documentclass[gray, 12pt]{beamer}
\usepackage{bussproofs}
\usepackage{listings,fancyvrb}
\usepackage{verbatimbox}
%% \usepackage{logicthemelive}
\author{Gyorgy Farkas}
\title{Types for data processing}
\begin{document}
\maketitle

\frame{
  \frametitle{Typing data processing seems compilcated}
  \begin{itemize}
    \item Data schemas give us a static view
    \item Reasoning about processing behavior is not supported by declared schemas
    \item programming languages are not very good at dealing with external data schemas as types
  \end{itemize}
}

\frame{
\begin{itemize}
  \item Do not want to deal with schemas explicitly
  \item Let us reason about data processing programs
\end{itemize}
}

\frame {
  \frametitle{Existing typings}
  \begin{itemize}
    \item Dynamic typing
    \item Relational Mapping
    \item Structural types
    \item Monadic wrappers for isolating behavior
    \item Type providers
    \item Full dependent types
  \end{itemize}
}

\frame{
  \frametitle{Suggested options}
  \begin{itemize}
    \item Extensible records, row types, enables dealing with data without predifined schema,
    while it remains statically typed

    \item Substructural types, reasoning about resource usage
  \end{itemize}
}

\frame{
  \frametitle{Extensible records}
  The extensible row datatype is actually a glorified Map from fields to values, that allows the following operations:
  \begin{equation*}
  \forall (r:\rho) (l, v:\tau)  (>> (l) (v:\tau) (r:\rho)) \rightarrow \{ (l, v:\tau) | r:\rho \} \label{E: extend} \tag{extend}
  \end{equation*}

  \begin{equation*}
   \forall (r:\rho) (l, v:\tau)  (<< (l) \{ (l, v:\tau) | r:\rho \} ) \rightarrow r:\rho \label{E: restrict} \tag{restrict}
  \end{equation*}
  \begin{equation*}
   \forall (r:\rho) (l, v:\tau)  (. (l) (\{ (l, v:\tau) | r:\rho \})) \rightarrow v:\tau \label{E: project} \tag{project}
   \end{equation*}
}

\frame{
  \frametitle{Type inference for rows}
  \begin{prooftree}
      \AxiomC{$s \sim  \{(l::\tau)|s'\} : \theta_1$}
       \AxiomC{$\theta_1\tau \sim \theta_1\tau':\theta_2$}
       \AxiomC{$\theta_2(\theta_1 s) \sim \theta_2(\theta_1 r):\theta_3$}
       \TrinaryInfC{ $ \{(l:: \tau)| r\} \sim s : \theta_1 \circ \theta_2 \circ \theta_3$ }
       \RightLabel{\scriptsize{uni-row}}
   \end{prooftree}

   \begin{prooftree}
     \AxiomC{$l \neq l'$}
     \AxiomC{$r \sim  \{(l::\tau)| r' \}:\theta$}
     \BinaryInfC{ $ \{(l':: \tau')| r\} \sim \{(l::\tau), (l'::\tau')| r' \}$ }
     \RightLabel{\scriptsize{row-swap}}
    \end{prooftree}

    \begin{prooftree}
       \AxiomC{ $ \{(l::\tau)| r\} \sim  \{(l::\tau)| r\}  : \emptyset$ }
       \RightLabel{\scriptsize{rowhead-row}}
    \end{prooftree}

      \begin{prooftree}
        \AxiomC{$fresh(\alpha)$}
         \AxiomC{$fresh(\beta)$}
         \BinaryInfC{ $\alpha \sim \{l::\beta | \alpha \} : \lbrack \alpha \rightarrow \{(l::\beta) | \alpha \}$ }
         \RightLabel{\scriptsize{row-var}}
      \end{prooftree}
}

\frame{
  \frametitle{cvsScheme Demo}
  Marketing is king
}

\frame{
  \frametitle{Substructural types - structural laws}
  \begin{itemize}
    \item Weakening
    \item Exchange
    \item Contraction
  \end{itemize}
}
\frame{
  \frametitle{Linear types}
  \begin{itemize}
    \item No Weakening or contraction, exchange might be ok.
    \item Exactly once semantics
  \end{itemize}
}


\frame{
  \frametitle{references}
  \begin{thebibliography}{9}
    \bibitem{wadler96} Nick Benton, Philip Wadler. (1996) Linear logic, monads, and the lambda calculus, 11. IEEE Symposium on Logic in Computer Science
    \bibitem{wadler95} David N. Turner, Philip Wadler, Christian Mossin.,(1995) Once upon a type 7. ICFP
    \bibitem{TAPL} Benjamin C. Pierce, (2002) Types and programming languages
    \bibitem{ATAPL} Benjamin C. Pierce, (2004) (ed.) Advanced Topics in Types and Programming Languages
    \bibitem{pfenning} Frank Pfenning, Linear Logic lectures, (2012) https://www.cs.cmu.edu/~fp/courses/15816-s12/
    \bibitem{leijen} Daan Leijen, (2005) \emph{Extensible records with scoped labels}, https://www.microsoft.com/en-us/research/wp-content/uploads/2016/02/scopedlabels.pdf
  \end{thebibliography}
}\end{document}
