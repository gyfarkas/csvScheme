\documentclass[gray, 10pt]{beamer}
\usepackage{bussproofs}
\usepackage{listings,fancyvrb}
\usepackage{verbatimbox}
\usepackage{amssymb}
%% \usepackage{logicthemelive}
\author{Gyorgy Farkas}
\title{Types for data processing}
\begin{document}
\maketitle
\frame{
  \frametitle{types, remarks}
\begin{itemize}
  \item types are for reasoning about program correctness vs. labelling storage locations
  \item ie. well typed programs do run with the expected semantics (soundness)
  \item no completness, not all possible correct program will be typeble
  \item Curry-Howard types are propositions
\end{itemize}
}
\frame{
  \frametitle{Typing data processing seems compilcated}
  \begin{itemize}
    \item Data schemas give us a static view
    \item programming languages are not very good at dealing with external data schemas as types
    \item Reasoning about processing behavior is not supported by declared schemas
    \item There are important behaviors not expressed (for exapmle: Spark narrow vs wide transformations)
  \end{itemize}
}

\frame {
  \frametitle{Existing typings}
  \begin{itemize}
    \item Dynamic typing (no typing)
    \item Relational Mapping (ORMs)
    \item Structural types (half typing)
    \item Monadic wrappers for isolating behavior
    \item Language integrated query
    \item Type providers
    \item Full dependent types
  \end{itemize}
}

\frame{
  \frametitle{Suggested options}
  \begin{itemize}
    \item Extensible records, row types, enables dealing with data without predifined schema,
    while it remains statically typed

    \item Substructural types, reasoning about resource usage
  \end{itemize}
}

\frame{
  \frametitle{Extensible records: operations}
  The extensible row datatype is actually a glorified Map from fields to values, that allows the following operations:
  \begin{equation*}
  \forall (r:\rho) (l :: (v:\tau))  (\gg l  ,(v:\tau), (r:\rho)) \rightarrow \{ (l :: (v:\tau)) | r:\rho \} \label{E: extend} \tag{extend}
  \end{equation*}

  \begin{equation*}
   \forall (r:\rho) (l :: (v:\tau)) (\ll l, \{ (l :: (v:\tau)) | r:\rho \} ) \rightarrow r:\rho \label{E: restrict} \tag{restrict}
  \end{equation*}
  \begin{equation*}
   \forall (r:\rho) (l ::(v:\tau))  (. l, (\{ l :: (v:\tau) | r:\rho \})) \rightarrow v:\tau \label{E: project} \tag{project}
   \end{equation*}
}

\frame{
  \frametitle{Type inference for rows}
  Unification
  \begin{prooftree}
      \AxiomC{$s \cong  \{(l::\tau)|s'\} : \theta_1$}
      \noLine
      \UnaryInfC{$\theta_1\tau \sim \theta_1\tau':\theta_2$}

      \AxiomC{$tail(r) \notin dom(\theta_1)$}
      \noLine
      \UnaryInfC{$\theta_2(\theta_1 s) \sim \theta_2(\theta_1 r):\theta_3$}
      \BinaryInfC{ $ \{(l:: \tau)| r\} \sim s : \theta_1 \circ \theta_2 \circ \theta_3$ }
       \RightLabel{\scriptsize{uni-row}}
   \end{prooftree}
 Rewrite rules
   \begin{prooftree}
     \AxiomC{$l \neq l'$}
     \AxiomC{$r \cong  \{(l::\tau)| r' \}:\theta$}
     \BinaryInfC{ $ \{(l':: \tau')| r\} \cong \{(l::\tau), (l'::\tau')| r' \}$ }
     \RightLabel{\scriptsize{row-swap}}
    \end{prooftree}

    \begin{prooftree}
       \AxiomC{ $ \{(l::\tau)| r\} \cong \{(l::\tau)| r\}  : \emptyset$ }
       \RightLabel{\scriptsize{rowhead-row}}
    \end{prooftree}

      \begin{prooftree}
        \AxiomC{$fresh(\alpha)$}
         \AxiomC{$fresh(\beta)$}
         \BinaryInfC{ $\gamma \cong \{l::\beta | \alpha \} : \lbrack \gamma \rightarrow \{(l::\beta) | \alpha \}\rbrack$ }
         \RightLabel{\scriptsize{row-var}}
      \end{prooftree}
}

\frame{
  \frametitle{cvsScheme Demo}
  Marketing is king
}

\frame{
  \frametitle{Substructural types - structural laws}
  \begin{itemize}
    \item Weakening
    \begin{prooftree}
      \AxiomC{$\Gamma \vdash A$}
      \UnaryInfC{$\Gamma, B \vdash A$}
    \end{prooftree}
    \item Exchange
    \begin{prooftree}
      \AxiomC{$A,B \vdash C$}
      \UnaryInfC{$B,A \vdash C$}
    \end{prooftree}
    \item Contraction
    \begin{prooftree}
      \AxiomC{$\Gamma, \Gamma, A \vdash B$}
      \UnaryInfC{$\Gamma,A \vdash B$}
    \end{prooftree}
  \end{itemize}
}
\frame{
  \frametitle{Linear types: new operators}
  \begin{itemize}
    \item No Weakening or contraction, exchange might be ok.
    \item Exactly once semantics
  \end{itemize}

  \begin{itemize}
    \item new operators:$!A$:"of course $A$"
    \item $A \otimes B$: "both $A$ and $B$",
    \item $A \multimap B$: "consume $A$ to get a $B$",
    \item  $A \And B$: "choose from $A$ and $B$"
  \end{itemize}
}
\frame{
  \frametitle{Linear logic: Some new rules}
    \begin{prooftree}
      \AxiomC{$\Gamma \vdash B, \Delta$}
      \AxiomC{$\Gamma' \vdash C, \Delta'$}
      \BinaryInfC{$\Gamma,\Gamma' \vdash B \otimes C, \Delta, \Delta'$}
    \end{prooftree}
    \begin{prooftree}
      \AxiomC{$\Gamma, A,B \vdash \Delta$}
      \UnaryInfC{$\Gamma, A \otimes B \vdash \Delta$}
    \end{prooftree}
    \begin{prooftree}
      \AxiomC{$\Gamma \vdash A, \Delta$}
      \AxiomC{$\Gamma' \vdash B, \Delta'$}
      \BinaryInfC{$\Gamma, \Gamma', A \multimap B  \vdash \Delta, \Delta'$}
    \end{prooftree}
    \begin{prooftree}
      \AxiomC{$\Gamma \vdash A, \Delta$}
      \UnaryInfC{$\Gamma \vdash A \multimap B, \Delta$}
    \end{prooftree}
}


\frame{
  \frametitle{references}
  \begin{thebibliography}{9}
    \bibitem{wadler96} Nick Benton, Philip Wadler. (1996) Linear logic, monads, and the lambda calculus, 11. IEEE Symposium on Logic in Computer Science
    \bibitem{wadler95} David N. Turner, Philip Wadler, Christian Mossin.,(1995) Once upon a type 7. ICFP
    \bibitem{TAPL} Benjamin C. Pierce, (2002) Types and programming languages
    \bibitem{ATAPL} Benjamin C. Pierce, (2004) (ed.) Advanced Topics in Types and Programming Languages
    \bibitem{pfenning} Frank Pfenning, Linear Logic lectures, (2012) https://www.cs.cmu.edu/~fp/courses/15816-s12/
    \bibitem{leijen} Daan Leijen, (2005) \emph{Extensible records with scoped labels}, https://www.microsoft.com/en-us/research/wp-content/uploads/2016/02/scopedlabels.pdf
  \end{thebibliography}
}\end{document}
